\begin{abstract}
Stencil codes are a common class of parallel application which occur across many diciplines of simulation and modelling.
This pattern of computation is ineherently parallelizable, yet its low arithmetic intensity hampers performance on conventional hardware.
This paper introduces CASTLE, an architectural pattern for implementing stencil codes in hardware.
CASTLE allows developers to take advantage of the massive parallelism and memory bandwiths offered by FPGAs.

\end{abstract}
